\section{Conclusiones}

Nosotros encaramos el trabajo en paralelo, primero dividiendo las ramas del diagrama de objetivos, y al mismo tiempo encarando el de contexto. Nos pareció razonable dejar los escenarios informales para el final, ya que tiene que estar todo lo demás para que los escenarios tengan sentido.

El TP nos resultó complicado por ser muy amplio, sobre todo en cuanto al diagrama de objetivos. Necesitamos iterar dos o tres veces para llegar a algo modularizado y legible. Nos sirvió la sugerencia de tratar de encontrar objetivos intermedios que agruparan objetivos de un y-refinamiento grande, por ejemplo, para organizar nuestra forma de pensar.

Por otra parte, a pesar de que en un principio dividimos las ramas de Objetivos para trabajar, después hubo que centralizar todo y revisar cada rama en función de las demás; eso fue un trabajo pesado en el que ramas enteras desaparecieron y muchas otras se acortaron. Por poner un elemplo, el sistema de alertas y notificaciones dejó de ser una parte separada para pasar a diluirse en requerimientos específicos para el sistema en distintas partes.

También, a la hora de listar los requerimientos del sistema, nos dimos cuenta de errores en el diagrama de objetivos (menores) que corregimos en función de los requisitos adicionales que necesitábamos para el sistema.

En el diagrama de contexto también costó manejar la gran cantidad de datos que tuvimos que plasmar. Había ciertas interacciones para las cuales no estaba claro cuál es el nivel de detalle ideal a la hora de representarlas en el diagrama. En particular, durante el transcurso del TP nosotros cambiamos la manera de interacción del cliente y del proveedor con el sistema (usando links de carga para no tener que pedir al otro agente que se logueara para cualquier interacción, y que por ende interactuar le resultara más sencillo).

Con los diagramas en la mano fue muy ágil construir los escenarios informales, ya que solamente había que seguir el flujo del proyecto desde objetivos, y revisar las interacciones.
