\section{Introducción}
En el siguiente documento se presenta el análisis preliminar de la implementación de un sistema de gestión de proyectos para la empresa DC Construcciones, para reemplazar el sistema manual actual, que al no escalar dificulta el manejo de muchos proyectos a medida que la empresa crece.

Este sistema debe ser capaz de resolver los problemas de seguimiento de los proyectos en todas sus fases, desde el momento en el que algún agente presenta un pre-proyecto, pasando por su creación y desarrollo (el período en el que se realiza efectivamente la obra) hasta el momento en que la obra se termina y se recopila información con el objetivo de mejorar los mecanismos de asignación de agentes a trabajos.

Para este fin mostramos:
\begin{itemize}
	\item Un diagrama de objetivos que explica de qué manera el software puede satisfacer las necesidades requeridas y analizando alternativas en función de diferentes objetivos blandos.
	\item Los requerimientos de sistema en función de los objetivos planteados.
	\item Un diagrama de contexto, donde se detallarán los límites entre el software a construir y los demás agentes. \item Una lista de posibles escenarios de uso para mostrar, con una selección representativa de situaciones hipotéticas que ilustran cómo actúa la solución planteada en la vida real.
\end{itemize}
