\subsection{Modelo conceptual}

Mostramos ahora nuestro modelo conceptual del sistema de la empresa. 

\begin{figure}[H]
\includegraphics[width=\linewidth]{diag/nuevos/concept2.png}
\end{figure}
\begin{figure}[H]
\includegraphics[width=\linewidth]{diag/nuevos/concept1.png}
\end{figure}

En el mismo puede verse que la clase \textit{Proyecto} tiene un atributo \textit{Status} 
que es un enumerado, con los estados posibles. Estos estados no son exactamente los 
mismos que se enuncian en el FSM de flujo general del proyecto (figura \ref{fsm-proj}), sino más 
bien son un subconjunto que creemos apropiado. 

La idea es que ese \textit{Status} 
sea lo que los usuarios del sistema vean como estado del proyecto en la interfaz. 
Usamos ese atributo además para predicar distintas cosas por OCL, por ejemplo 
qué cosas debe tener cargadas un proyecto en determinada etapa. 

\subsubsection{Condiciones OCL}
\begin{itemize}
		\item	\textbf{Momentos en los cuales se puede redefinir el PM o proveedor de un proyecto:}
	
			Context: Proyecto
			
			\begin{tabular}{ll}
				self.redefiniendoProveedor $\Rightarrow$	& BuscandoProveedor < self.Status $\leq$ ObraEnCurso	\\
				self.redefiniendoPM $\Rightarrow$			& EligiendoPM < self.Status $\leq$ ObraEnCurso			\\
			\end{tabular}

	\item \textbf{Estado del proyecto:}

			Context: Proyecto
			
			\begin{tabular}{ll}
				\laterThan{EligiendoPM}				& \notEmpty{self.historicoPM} and	\\
													& ((self.redefiniendoPM) xor \\
													& (self.historicoPM\applyParam{exists}{pm$|$pm.Actual}))	\\
				
				and \laterThan{DefiniendoAlcance}	& \notEmpty{self.Alcances} \\
				
				and \laterThan{BuscandoProveedor}	& self.Alcances\applyParam{select}{a$|$a.Actual}.Seleccionada\notEmpty{}	\\
				
				and \laterThan{FirmandoContratos}	& \notEmpty{contratoCliente(self, self.Solicitante)} and	\\
													& (self.redefiniendoProveedor xor							\\
													& self.Alcances\applyParam{select}{a$|$a.Actual}.ContratoProveedor	\\
													& \notEmpty{})	\\
				
				and \laterThan{ConsiguiendoFeedback}	& self.HistoricoPM\applyParam{forAll}{a$|$
				a.\notEmpty{Feedback}} and	\\
														& self.Alcances\applyParam{collect}{AsignacionProveedor} \\
														& \applyParam{forAll}{a$|$a.Feedback\notEmpty{}} \\
			\end{tabular}
			
	\item \textbf{Seguro de caución al día para proyectos actuales:}
	
			Context: Alcance Proyecto
			
			self.Actual $\Rightarrow$ (\notEmpty{self.Seleccionada.Proveedor.Seguro} \\
			and self.Seleccionada.Proveedor.Seguro.Vencimiento	\\
			$\geq$ self.ContratoProveedor.FechaFin)	\\
			
			%(self.Actual and self.Proyecto.Status $\in$ [BuscandoProveedor, obraEnCurso]	\\
			%and \notEmpty{self.PropuestaProveedor.contratoProveedor}) 		\\
			%$\Rightarrow$ (\notEmpty{self.Proveedor.Seguro} and self.Proveedor.Seguro.Vencimiento	\\
			%> self.PropuestaProveedor.contratoProveedor.FechaFin)	\\
			
	\item \textbf{Los puntajes de los agentes se corresponden con los puntajes seg\'un proyectos:}
			
			Context: PM
			
			self.puntaje ==	self.Asignaciones\applyParam{collect}{FeedbackPM}\applyParam{collect}{Calificacion}\apply{average}
			\footnote{Consideramos que average de vacío da 0.}
			
			Context: Proveedor
			
			self.puntaje ==	self.Asignaciones\applyParam{collect}{FeedbackProv}\applyParam{collect}{Calificacion}\apply{average}
			\footnote{Idem nota anterior}
			
	
	\item \textbf{Las asignaciones no se pisan en tiempo}
	
			Context: Proyecto
			
			self.Alcances\applyParam{collect}{AsignacionProveedor}\applyParam{forAll}{a1 $\neq$ a2$\|$a1.Inicio > a2.Fin or a2.Inicio > a1.Fin}
	
	\item \textbf{Hay a lo sumo un alcance actual y una asignación de PM actual}
	
			Context: Proyecto
			
			self.Alcances\applyParam{select}{a$\|$a.Actual}\apply{size} $\leq$ 1
			
			self.HistoricoPM\applyParam{select}{a$\|$a.Actual}\apply{size} $\leq$ 1
			
			self.Status > ObraEnCurso $\Rightarrow$ self.Alcances\applyParam{select}{a$\|$a.Actual}\apply{size} = 0
			
	\item \textbf{Ciclo Proveedor - Asignaci\'on - Propuesta}
	
			Context: Asignación Proveedor
			
			self.Proveedor == self.Seleccionada.Proveedor
	
	%\item \textbf{Los estados del proyecto pero para el otro lado?}
	%\item \textbf{Fechas de los reportes?}
	%\item \textbf{Solapamiento de PMs historico}
	%\item \textbf{Solapamiento de proveedores historico?}
	%\item \textbf{Porcentaje de las comisiones}
	%\item \textbf{Fechas de fin luego de fechas de inicio}
	
\end{itemize}
