\section{Flujo Global del Sistema y Presunciones de Dominio}

En esta sección contaremos resumidamente cómo entendemos que funciona 
el flujo completo de la empresa a la hora de encarar un nuevo proyecto, 
desde el inicio en que es solicitado, hasta que se pone en marcha y finalmente 
se termina. A la vez en la siguiente descripción, mencionaremos algunos detalles 
de alto nivel de las soluciones que planteamos para el sistema en cuestión. 

\begin{enumerate}
    \item Primero un Cliente contacta a la empresa por un nuevo trabajo, 
    a través de un corto formulario en una interfaz web del sistema. 
    
    \item Al hacer esto, los Gerentes son notificados, quienes pueden ver 
    estas solicitudes en el sistema. Si lo creen pertinente, crean entonces 
    un nuevo proyecto en el sistema con los datos necesarios (potencialmente 
    corrigiendo/clarificando datos ingresados por el Cliente). El Cliente 
    es notificado cuando el proyecto es finalmente creado, a modo de confirmación. 
    
    \item Paralelamente, los Administradores son notificados. Ellos tendrán la 
    tarea de garantizar si se trata de un Cliente nuevo o no, y conseguir los 
    datos necesarios de ser ese caso. 
    
    \item Luego de creado el Proyecto, el Gerente puede consultar el estado 
    de los PMs, filtrar/ordernar la lista a gusto, y en algún momento, 
    asignar un PM al proyecto. 
    
    \item Cuando esto sucede, el PM es notificado y debe ponerse a trabajar. 
    Su primer tarea es definir el primer alcance del Proyecto. Para esto, 
    dialogará con el Cliente (por o fuera del sistema) hasta llegar a un acuerdo. 
    El PM carga este alcance en el sistema y el Cliente es notificado de esto. 
    El Cliente luego aprueba este alcance, o lo rechaza indicando por qué, y 
    se repite el ciclo hasta llegar a un alcance aprobado. 
    
    \item Con el primer alcance definido, el PM se pone a buscar Proveedores 
    acordes en la base de datos, a envía a quienes crea pertinentes una solicitud 
    de propuesta. Estos proveedores son notificados por correo. Al momento de filtrar 
    en el sistema, el mismo informará si los Proveedores tienen seguro de caución apto 
    para la fecha estimada de finalización del proyecto. 
    
    \item Los proveedores notificados podrán llenar sus propuestas mediante un 
    formulario en una interfaz web con un link privado autogenerado para ellos. 
    
    \item Eventualmente, con las propuestas recibidas, el PM seleccionará la más acorde 
    y el dicho Proveedor será notificado de que su propuesta fue la aceptada. 
    El PM será alertado por el sistema si pasa demasiado tiempo sin llegar a una 
    decisión de propuesta, así como también será notificado por nuevas propuestas 
    que lleguen. También podrá añadir propuestas manualmente por los Proveedores en 
    caso de eventualidades. 
    
    \item Con la propuesta seleccionada, el PM pasará a armar los contratos, 
    tanto para el Proveedor como el Cliente. Aquí será importante verificar que 
    el seguro de caución del Proveedor no vencerá hasta finalizado el contrato con 
    el mismo. Una vez armados los contratos, se envían a los Gerentes para revisión. 
    
    \item Los Gerentes son notificados de los contratos, y podrán aprobarlos o 
    rechazarlos explicando por qué (en cuyo caso, el PM los corrige y vuelve a mandarlos). 
    Finalmente, en algún momento ambos contratos son aprobados, el PM es notificado, 
    y automáticamente pueden ser enviados al Cliente y Proveedor. 
    
    \item Al Cliente y Proveedor les llega su contrato, y pueden firmarlos y enviarlo 
    mediante un link especial ellos mismos. También pueden enviarlo por mail de ser 
    necesario, o pueden firmarlo en papel y entregarselo al PM en mano. 
    Tanto el PM como el Cliente y el Proveedor son alertados si pasa demasiado tiempo 
    sin que los contratos hayan sido subidos firmados. 
    
    \item Una vez recibidos los contratos firmados, el PM asigna al Proveedor al 
    Proyecto, y la obra puede comenzar junto con el seguimiento. 
    
    \item El PM periódicamente deberá enviar reportes de la obra que serán notificados 
    a los Gerentes y al Cliente. Asímismo, el Cliente podrá reportar problemas 
    con la obra mediante una interfaz web, o mismo por correo. 
    
    \item Finalmente, la obra se termina, el PM marca esto en el sistema y las partes 
    son notificadas. Al momento que hace esto, el sistema genera los formularios 
    de feedback para el Cliente (feedback sobre los PMs) y para los PMs (feedback sobre 
    los proveedores). 
    
    \item En cualquier momento, el Gerente puede cambiar el PM del Proyecto si 
    lo cree necesario. 
    
    \item Así también, puede requerirse cambiar de Proveedor. En caso de hacer 
    esto, se definirá un nuevo alcance y se empezará nuevamente todo el ciclo 
    de búsqueda de Proveedor, firma de contratos y etc. El nuevo alcance 
    podría ser distinto al anterior, dado el trabajo que haya llegado a 
    hacer el Proveedor saliente, o porque el Cliente solicitó adicionales. 
\end{enumerate}
