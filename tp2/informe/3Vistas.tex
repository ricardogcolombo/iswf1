\section{Vistas}

\subsection{Casos de Uso}
	\subsubsection{Diagrama de Casos de Uso}
En esta sección intentaremos terminar de definir de manera mas precisa y detallada las diferentes interacciones que puedan existir entre nuestro sistema y los diferentes actores. Para ello utilizaremos el modelado con un diagrama de casos de uso.

\begin{figure}[H]
\centering
\includegraphics[width=\linewidth]{diag/viejos/cu1.pdf}
\caption{Casos de Uso: Cliente y Subsistema de Mail}
\label{cu1}
\end{figure}

\begin{figure}[H]
\centering
\includegraphics[width=\linewidth]{diag/viejos/cu2.pdf}
\caption{Casos de Uso: PM y Subsistema de Mail}
\label{cu2}
\end{figure}

\begin{figure}[H]
\centering
\includegraphics[width=\linewidth]{diag/viejos/cu3.pdf}
\caption{Casos de Uso: Subsistema de Mail}
\label{cu3}
\end{figure}

\begin{figure}[H]
\centering
\includegraphics[width=\linewidth]{diag/viejos/cu4.pdf}
\caption{Casos de Uso: Gerente y Subsistema de Mail}
\label{cu4}
\end{figure}

\begin{figure}[H]
\centering
\includegraphics[width=\linewidth]{diag/viejos/cu5.pdf}
\caption{Casos de Uso: Proveedor y Subsistema de Mail}
\label{cu5}
\end{figure}

\begin{figure}[H]
\centering
\includegraphics[width=\linewidth]{diag/viejos/cu6.pdf}
\caption{Casos de Uso: Administrador y Subsistema de Mail}
\label{cu6}
\end{figure}

Para finalizar detallaremos los distintos casos de uso en los que ademas 
intentaremos ver cuales son los comportamientos alternativos que pueda 
tener nuestro sistema en las distintas circunstancias.

% ADMIN
		\subsubsection{Detalles de Casos de Uso del Administrador}
\begin{longtable}{| p{.60\textwidth} | p{.40\textwidth} |} 
    \hline
    \multicolumn{2}{|p{16cm}|}{
        \textbf{Caso de uso:} Consultando Proveedores \newline
        \textbf{Actor:} Administrador\newline
        \textbf{Pre:}  True\newline
        \textbf{Post:}El Administrador consulta el proveedor.
    }\\
    \hline
    1.El sistema le solicita que ingrese los filtros de busqueda  & \\
    \hline
    2.El Administrador Agrega los datos del provedor que esta buscando& \\
    \hline
    3.El Sistema encuentra el proveedor y muestra los datos de contacto & 3.1.El proveedor solicitado no se encuentra en el sistema \newline 3.2 Fin de C.U.  \\
    \hline
    4.El Administrador decide elimiar el proveedor. Extiende Caso de uso Borrando Proveedor.& \\
    \hline
    5.El Administrador decide agregar el seguro de Caucion de proveedor. Extiende Caso de uso Agregando Seguro de Caucion.& \\
    \hline
    6.El Administrador decide consultar estado del seguro de Caucion de proveedor. Extiende Caso de uso Consultando estado de seguro de caucion.& \\
    \hline
    7.El Administrador decide actualizar datos del proveedor. Extiende Caso de uso Actualizando datos de proveedor.& \\
    \hline
    8.Fin de C.U.& \\
    \hline
\end{longtable}


\begin{longtable}{| p{.60\textwidth} | p{.40\textwidth} |} 
    \hline
    \multicolumn{2}{|p{16cm}|}{
        \textbf{Caso de uso:} Consultando Clientes \newline
        \textbf{Actor:} Administrador\newline
        \textbf{Pre:}  True\newline
        \textbf{Post:}El Administrador Consulta los clientes.
    }\\
    \hline
    1.El sistema le solicita que ingrese los filtros de busqueda  & \\
    \hline
    2.El Administrador Agrega los datos del cliente que esta buscando& \\
    \hline
    3.El Sistema encuentra el cliente y muestra los datos de contacto & 3.1.El cliente solicitado no se encuentra en el sistema \newline 3.2 Fin de C.U.  \\
    \hline
    4.El Administrador decide elimiar el clientes. Extiende Caso de uso Borrando cliente.& \\
    \hline
    5.El Administrador decide actualizar datos del cliente. Extiende Caso de uso Actualizando datos cliente.& \\
    \hline
    6.El Administrador Agregar un nuevo cliente. Extiende Caso de uso Agregando nuevo Cliente.& \\
    \hline
    7.El Administrador decide Solicitar Datos al cliente. Extiende Caso de uso Solicitando datos al Cliente.& \\
    \hline
    8.Fin de C.U.& \\
    \hline
\end{longtable}

\begin{longtable}{| p{.60\textwidth} | p{.40\textwidth} |} 
    \hline
    \multicolumn{2}{|p{16cm}|}{
        \textbf{Caso de uso:} Agregando Proveedor \newline
        \textbf{Actor:} Administrador\newline
        \textbf{Pre:}  True\newline
        \textbf{Post:} El proveedor fue agregado al sistema
    }\\
    \hline
    1.El sistema le solicita que ingrese los datos de contacto del proveedor & \\
    \hline
    2.El Administrador Agrega los datos del provedor como son nombres, datos de contacto y datos relacionados al negocio` &  \\
    \hline
    3.El Sistema valida los datos de contacto para ver si no se encuentra registrado & 3.1.El proveedor ya esta dado de alta \newline 3.2 Fin de C.U.  \\
    \hline
    4.El Sistema Pregunta si desea Agregar el seguro de caucion&\\
    \hline
    5.El Administrador Agrega Seguro de caucion, Extiende Caso de Uso Agregar seguro de Caucion. & 5.1 El Administrador decide agregarlo luego. Continua en paso 5 \\
    \hline
    6.El Sistema Guarda el proveedor. USA Notificando adicion a la base de proveedores& \\
    \hline
    7.Fin de C.U.& \\
    \hline
\end{longtable}

\begin{longtable}{| p{.60\textwidth} | p{.40\textwidth} |} 
    \hline
    \multicolumn{2}{|p{16cm}|}{
        \textbf{Caso de uso:} Borrando Proveedor \newline
        \textbf{Actor:} Administrador\newline
        \textbf{Pre:}  Proveedor Seleccionado\newline
        \textbf{Post:} El proveedor fue eliminado del sistema
    }\\
    \hline
    1.El sistema  muetra las opciones para realizar con un proveedor & \\
    \hline
    2.El Administrador selecciona eliminar& \\
    \hline
    3.El Sistema lanza un mensaje consultando si desea eliminar el proveedor &  \\
    \hline
    4.El Administrador selecciona que SI desea eliminar el proveedor & 4.1.1 El Sistema nota que el proveedor sigue asignado a un proyecto en curso, muestra un mensaje por pantalla notificando este problema \newline 4.1.2 Fin Caso de Uso \newline 4.2.1 El usuario Selecciona que NO desea eliminar el proveedor \newline4.2.2 Fin de Caso de uso\\
    \hline
    5.El Sistema elimina el proveedor del sistema &  \\
    \hline
    6.Fin de C.U.& \\
    \hline
\end{longtable}
\begin{longtable}{|p{.60\textwidth}|p{.40\textwidth}|}
    \hline
    \multicolumn{2}{|p{16cm}|}{
        \textbf{Caso de uso:} Actualizando datos de contacto de Proveedor \newline
        \textbf{Actor:} Administrador\newline
        \textbf{Pre:}  Proveedor seleccionado\newline
        \textbf{Post:} El Administrador actualiza los datos del Proveedor
    }\\
    \hline
    1.El sistema  muetra las opciones para realizar con un proveedor & \\
    \hline
    2.El Administrador selecciona Actualizar Datos Proveedor&   \\
    \hline
    3.El Sistema muestra todos los campos con los datos del proveedor para modificar y dos botones , uno para guardar y otro para cancelar&  \\
    \hline
    4.El Administrador Modifica los datos y toca salvar & 4.1.El Administrador toca cancelar \newline 4.2 Fin Caso de Uso \\
    \hline
    5.El Sistema guarda los cambios al proveedor &  \\
    \hline
    6.Fin de C.U.& \\
    \hline
\end{longtable}

\begin{longtable}{|p{.60\textwidth}|p{.40\textwidth}|}
    \hline
    \multicolumn{2}{|p{16cm}|}{
        \textbf{Caso de uso:} Consultando estado de seguro de caucion \newline
        \textbf{Actor:} Administrador\newline
        \textbf{Pre: }  Proveedor seleccionado \newline
        \textbf{Post:} El Administrador consulta estado del seguro de caucion
    }\\
    \hline
    1.El Administrador selecciona la opcion consultar estado de seguro de caucion&  \\
    \hline
    2.El Sistema muestra el estado de seguro de caucion&  \\
    \hline
    3.Fin de C.U.& \\
    \hline
\end{longtable}

\begin{longtable}{|p{.60\textwidth}|p{.40\textwidth}|}
    \hline
    \multicolumn{2}{|p{16cm}|}{
        \textbf{Caso de uso:} Agregando seguro de caucion\newline
        \textbf{Actor:} Administrador\newline
        \textbf{Pre: }  Proveedor seleccionado\newline
        \textbf{Post:} El Administrador Agrega un nuevo seguro de caucion
    }\\
    \hline
    1.El Administrador selecciona la opcion agregar de seguro de caucion& \\
    \hline
    2.El Sistema muestra la opcion de ingreso de validez del seguro de caucion y el ingreso del archivo con el seguro de caucion&  \\
    \hline
    3.El Administrador ingresa la fecha de validez, agrega el archivo del escaneo del seguro de caucion y guarda los cambios&3.1 El administrador Apreta el boton cancelar \newline 3.2 Fin del Caso de uso \\
    \hline
    4.El Sistema verifica la fecha de validez y que no que no exista otro seguro de caucion & \\
    \hline
    5.El Sistema aprueba los datos ingresados y guarda los cambios &5.1.1 Los datos ingresados son incorrectos,el sistema muestra mensaje de error \newline 5.1.2 vuelve al paso 4 \newline 5.2.1 Existe otro seguro de caucion en la misma fecha, el sistema muestra un mensaje avisando que actualizara los datos con el nuevo seguro de caucion. \newline 5.2 vuelve al paso 4\\
    \hline
    6.Fin de C.U.& \\
    \hline
\end{longtable}

\begin{longtable}{|p{.60\textwidth}|p{.40\textwidth}|}
    \hline
    \multicolumn{2}{|p{16cm}|}{
        \textbf{Caso de uso:} Solicitando datos/seguro de caucion de proveedor\newline
        \textbf{Actor:} Administrador\newline
        \textbf{Pre:} Proveedor seleccionado\newline
        \textbf{Post:} El Administrador Agrega un nuevo seguro de caucion
    }\\
    \hline
    1.El sistema  muetra las opciones para realizar con un proveedor & \\
    \hline
    2.El Administrador selecciona la opcion agregar de seguro de caucion& \\
    \hline
    3.El Sistema muestra la opcion de ingreso de validez del seguro de caucion y el ingreso del archivo con el seguro de caucion&  \\
    \hline
    4.El Administrador ingresa la fecha de validez, agrega el archivo del escaneo del seguro de caucion y guarda los cambios&4.1 El administrador Apreta el boton cancelar \newline 4.2 Fin del Caso de uso \\
    \hline
    5.El Sistema verifica la fecha de validez y que no que no exista otro seguro de caucion & \\
    \hline
    6.El Sistema aprueba los datos ingresados y guarda los cambios &6.1 Los datos ingresados son incorrectos, o hay otro seguro de caucion en la misma fecha  \newline 6.2 vuelve al paso 4\\
    \hline
    7.Fin de C.U.& \\
    \hline
\end{longtable}


% cliente
		\subsubsection{Detalles de Casos de Uso del Cliente}
\begin{longtable}{|p{.60\textwidth}|p{.40\textwidth}|}
    \hline
    \multicolumn{2}{|p{16cm}|}{
        \textbf{Caso de uso:} Contactando por nuevos trabajos\newline
        \textbf{Actor:} Cliente\newline
        \textbf{Pre: }  True\newline
        \textbf{Post:} El Cliente deja el contacto para un nuevo trabajo en el sistema
    }\\
    \hline
    1.El Cliente seleccion la opcion de contacto y completa los datos de contacto&2.1 El Cliente no ingresa datos de contacto\newline 1.2 Fin caso de uso   \\
    \hline
    2.El Sistema notifica nuevo proyecto. USA Notificando nuevos trabajos a gerente& \\
    \hline
    3.Fin de C.U.& \\
    \hline
\end{longtable}


\begin{longtable}{|p{.60\textwidth}|p{.40\textwidth}|}
    \hline
    \multicolumn{2}{|p{16cm}|}{
        \textbf{Caso de uso:} Completando Feedback de PM\newline
        \textbf{Actor:} Cliente\newline
        \textbf{Pre: }El Cliente Accede al sistema con link Provisto en alta de proyecto\newline
        \textbf{Post: }El Cliente Completo la encuesta
    }\\
    \hline
    1.El Usuario accede al link que recibio en el Mail y le abre una pagina web con la encuesta a completar & 1.1.El Usuario desestima el mail .\newline 1.2Fin de C.U.\\
    \hline
    2.El Sistema Muestra un formulario con un Area de texto libre para completar sobre el feedback del PM &\\
    \hline 
    3.El Usuario Completa el formulario y Apreta la opcion de enviar el formulario& \\
    \hline
    4.El Sistema Registra la nueva encuesta&\\
    \hline
    5.El Sistema actualiza el puntaje del PM&\\
    \hline
    6.Fin del C.U&\\
    \hline
\end{longtable}

\begin{longtable}{|p{.60\textwidth}|p{.40\textwidth}|}
    \hline
    \multicolumn{2}{|p{16cm}|}{
        \textbf{Caso de uso:} Reportando Problemas\newline
        \textbf{Actor:} Cliente\newline
        \textbf{Pre: }El Cliente Accede al sistema con link Provisto en alta de proyecto\newline
        \textbf{Post: }El Cliente notifico de un problema
    }\\
    \hline
    1.El Usuario entra a la pagina web con el link provisto y selecciona la opcion reportar problema.&\\
    \hline
    2.El Sistema muestra un formulario con espacio para escribir un detalle del problema&    \\
    \hline
    3.El Cliente Completa dicho formulario y selecciona la opcion Enviar& \\
    \hline
    4.El Sistema Registra el problema. Extiende Notificando Problemas en Proyectos a gerentes y PM&\\
    \hline
    5.Fin del C.U&\\
    \hline
\end{longtable}

\begin{longtable}{|p{.60\textwidth}|p{.40\textwidth}|}
    \hline
    \multicolumn{2}{|p{16cm}|}{
        \textbf{Caso de uso:} Enviando Contrato firmado \newline
        \textbf{Actor:} Cliente\newline
        \textbf{Pre: }El Cliente Accede al sistema con link Provisto en alta de proyecto\newline
        \textbf{Post: }El Envio el contrato firmado
    }\\
    \hline
    1.El Usuario entra a la pagina web con el link provisto y selecciona la opcion enviar contrato.&\\
    \hline
    2.El Sistema pide que seleccion el archivo pdf firmado desde su computadora&    \\
    \hline
    3.El Cliente Selecciona el archivo.& \\
    \hline
    4.El Sistema Guarda el contrato firmado. Extiende Notificando contrato de Cliente enviado a PM&\\
    \hline
    5.Fin del C.U&\\
    \hline
\end{longtable}

\begin{longtable}{|p{.60\textwidth}|p{.40\textwidth}|}
    \hline
    \multicolumn{2}{|p{16cm}|}{
        \textbf{Caso de uso:} Respondiendo solicitud de Datos \newline
        \textbf{Actor:} Cliente\newline
        \textbf{Pre: }El Cliente Accede al sistema con link Provisto en mail de solicitud\newline
        \textbf{Post: }El Envio el contrato firmado
    }\\
    \hline
    1.El Usuario entra a la pagina web con el link provisto.&\\
    \hline
    2.El Sistema Presenta un formulario con datos obligatorios como telefono, ubicacion, nombre, datos del negocio, servicios y productos que ofrece &    \\
    \hline
    3.El Cliente Completa los datos y presiona el boton guardar. &3.1 El Cliente no completa los datos.\newline 3.2 El Sistema lo marca como incompleto.\newline 3.3 Fin del C.U.\\
    \hline
    4.El Sistema Guarda los datos. Extiende Notificando datos completados al Admin&\\
    \hline
    5.Fin del C.U&\\
    \hline
\end{longtable}


% PM
		\subsubsection{Detalles de Casos de Uso del PM}
\begin{longtable}{|p{.60\textwidth}|p{.40\textwidth}|}
    \hline
    \multicolumn{2}{|p{16cm}|}{
        \textbf{Caso de uso:}Consultando Proyectos\newline
        \textbf{Actor:} PM\newline
        \textbf{Pre: }El PM esta logueado en el sistema\newline
        \textbf{Post:} El PM ve todos los proyectos registrados en el sistema
    }\\
    \hline
    1.El PM selecciona buscar proyectos& \\
    \hline
    2.El sistema muestra todos los proyectos disponibles en forma de lista mostrando nombre del cliente y nombre del proyecto &\\
    \hline
    3.El PM decide filtrar por Cliente y seleccion la opcion de fitlrar por cliente& \\
    \hline
    4.El PM escribe el nombre del Cliente& \\
    \hline
    5.El Sistema lista todos los resultados de busqueda&\\
    \hline
    6.Fin del C.U.&\\
    \hline
\end{longtable}

\begin{longtable}{|p{.60\textwidth}|p{.40\textwidth}|}
    \hline
    \multicolumn{2}{|p{16cm}|}{
        \textbf{Caso de uso:}Consultando alcance de Proyectos\newline
        \textbf{Actor:} PM\newline
        \textbf{Pre: }Proyecto Seleccionado\newline
        \textbf{Post:} El PM Puede ve los alcances de un proyecto
    }\\
    \hline
    1.El Pm Busca un proyecto. Extiende Consultando Proyectos& \\
    \hline
    2.El PM selecciona ver los alcances del proyecto&  \\
    \hline
    3.El sistema muestra los alcances del proyectos& \\
    \hline
    4.Fin del C.U&\\
    \hline
\end{longtable}

\begin{longtable}{|p{.60\textwidth}|p{.40\textwidth}|}
    \hline
    \multicolumn{2}{|p{16cm}|}{
        \textbf{Caso de uso:}Consultando TOP de proveedores\newline
        \textbf{Actor:} PM\newline
        \textbf{Pre: }Proyecto Seleccionado\newline
        \textbf{Post:} El PM obtiene una lista de proveedores ordenados segun los filtros seleccionados
    }\\
    \hline
    1.El sistema muestra los filtros de busqueda como nombre de proveedor, datos del negocio o ubicacion. & \\
    \hline
    2.El PM completa los filtros de busqueda que necesita&  \\
    \hline
    3.El sistema muestra los resultados basado en los filtros de busqueda ordenados por algun criterio seleccionado& \\
    \hline
    4.El PM Selecciona Proveedor &\\
    \hline
    5.El PM Desea enviar una solucitud de presupuesto.Extiende Solictud de propuesta a proveedores &5.1 El PM Desea seguir viendo otros proveedores y presiona el voton retroceder \newline 5.2 Contiuna en el paso 3\\
    \hline
    6.Fin del C.U&\\
    \hline
\end{longtable}


\begin{longtable}{|p{.60\textwidth}|p{.40\textwidth}|}
    \hline
    \multicolumn{2}{|p{16cm}|}{
        \textbf{Caso de uso:}Consultando propuestas de los proveedores\newline
        \textbf{Actor:} PM\newline
        \textbf{Pre: }Proyecto Seleccionado\newline
        \textbf{Post:} El PM obtiene una lista de las propuestas presentadas por los proveedores
    }\\
    \hline
    1.El PM selecciona la opcion de ver propuestas presentadas&\\
    \hline
    2.El sistema muestra las propuestas presentadas por los distintos proveedores & \\
    \hline
    3.El PM selecciona una propuesta  &\\
    \hline
    4.El sistema muestra el detalle de la propuesta seleccionada sobre productos y costos.&4.1 El PM Desea imprimir la propuesta y selecciona boton Imprimir. \newline 4.2. El PM desea volver a la lista de propuestas. Continua en paso 2\\
    \hline
    5.El PM Desea Asignar el proveedor al proyecto. Extiende Asignando proveedor al proyecto.&\\
    \hline
    6.&\\
    \hline
    5.Fin del C.U&\\
    \hline
\end{longtable}

\begin{longtable}{|p{.60\textwidth}|p{.40\textwidth}|}
    \hline
    \multicolumn{2}{|p{16cm}|}{
        \textbf{Caso de uso:}Marcando una propuesta como seleccionada\newline
        \textbf{Actor:} PM\newline
        \textbf{Pre: }Proyecto Seleccionado\newline
        \textbf{Post:} El PM selecciona una de las propeustas para el proyecto seleccionado
    }\\
    \hline
    1.El PM selecciona una propuesta para el proyecto y guarda los cambios&\\
    \hline
    2.El sistema envia una notificacion al proeedor.USA Notificando seleccion de propuesta a proveedor& 2.1 El Proveedor no notifica como recibida la notificacion \newline 2.2 el Sistema no marca como seleccionada la propuesta\\
    \hline
    3.Fin del C.U&\\
    \hline
\end{longtable}

\begin{longtable}{|p{.60\textwidth}|p{.40\textwidth}|}
    \hline
    \multicolumn{2}{|p{16cm}|}{
        \textbf{Caso de uso:}Generando reporte de proyecto\newline
        \textbf{Actor:} PM\newline
        \textbf{Pre: }Proyecto Seleccionado\newline
        \textbf{Post:} El PM agrega un detalle de los avances en el proyecto
    }\\
    \hline
    1.El PM selecciona la opcion de agregar reporte &\\
    \hline
    2.El sistema muestra las un formulario para completar con detalles de tareas y fecha, ademas puede asignar un de estado como critico, con complicaciones o estable& \\
    \hline
    3.El PM completa un formulario con los avances del proyecto y guarda el reporte&\\
    \hline
    4.El sistema guarda el formulario&\\
    \hline
    5.Fin del C.U&\\
    \hline
\end{longtable}


\begin{longtable}{|p{.60\textwidth}|p{.40\textwidth}|}
    \hline
    \multicolumn{2}{|p{16cm}|}{
        \textbf{Caso de uso:}Enviando solicitud de propuesta a proveedores\newline
        \textbf{Actor:} PM\newline
        \textbf{Pre: }Proyecto Seleccionado\newline
        \textbf{Post:} El PM envia la soliicitud de propuestas a los mejores proveedores
    }\\
    \hline
    1.El sistema muestra la lista de TOP de proveedores & \\
    \hline
    2.El PM selecciona varios proveedores y selecciona la opcion de enviar solicitud de propuesta& \\
    \hline
    3.El sistema envia la notificacion.USA Notificando solicitud de porpuestas a proveedores& \\
    \hline
    4.Fin del C.U&\\
    \hline
\end{longtable}

\begin{longtable}{|p{.60\textwidth}|p{.40\textwidth}|}
    \hline
    \multicolumn{2}{|p{16cm}|}{
        \textbf{Caso de uso:}Armando Contrato del Proyecto \newline
        \textbf{Actor:} PM\newline
        \textbf{Pre: }Proyecto Seleccionado\newline
        \textbf{Post:} Arma y agrega contartos al proyecto
    }\\
    \hline
    1.El PM Selecciona Armar Contrato para el proyecto. Extiende Consultando Proyectos & \\
    \hline
    2.El Sistema muestra los templates de contratos de proyectos en el sistema& \\
    \hline
    3.El PM Selecciona un template de contrao& 3.1 El Cliente Desea elegir otro Template y selecciona el boton retornar a lista de contratos. \newline 3.2. Continua en paso 2 \\
    \hline
    4.El PM Completa el Template con los datos del proyecto como datos de Cliente, Proveedores y costos. Presiona Guardar Contrato en proyecto&\\
    \hline
    5.El Sistema Ofrece la opcion de imprimir contrato y guarda el contrato como parte del proyecto. USA Notificando Nuevos Contratos Para Aprobar a Gerentes&\\
    \hline
    6.El PM Apreta Aceptar e imprime el contrato.&\\
    \hline
    7.Fin del C.U. &\\
    \hline
\end{longtable}


\begin{longtable}{|p{.60\textwidth}|p{.40\textwidth}|}
    \hline
    \multicolumn{2}{|p{16cm}|}{
        \textbf{Caso de uso:}Cerrando Proyecto \newline
        \textbf{Actor:} PM\newline
        \textbf{Pre: }Proyecto Seleccionado\newline
        \textbf{Post:} Arma y agrega contartos al proyecto
    }\\
    \hline
    1.El PM Selecciona la opcion cerrar proyecto. Extiende Consultando Proyectos & \\
    \hline
    2.El Sistema Muestra un mensaje de confirmacion& \\
    \hline
    3. El PM Selecciona aceptar & 3.1 El PM Selecciona Cancelar \newline 3.2 Continua en paso 1\\
    \hline
    4.El Sistema marca el proyecto como cerrado. Extiende Notificando Cierre de proyecto y terminacion de obra a gerente &\\
    \hline
    5.Fin de C.U. &\\
    \hline
\end{longtable}

\begin{longtable}{|p{.60\textwidth}|p{.40\textwidth}|}
    \hline
    \multicolumn{2}{|p{16cm}|}{
        \textbf{Caso de uso:}Agregando Propuesta Proveedor Manualmente \newline
        \textbf{Actor:} PM\newline
        \textbf{Pre: }Proyecto Seleccionado\newline
        \textbf{Post:} Agrega una propuesta de proveedor a un proyecto
    }\\
    \hline
    1.El PM Selecciona la opcion Agregar Propuesta.& \\
    \hline
    2.El sistema abre una ventana de dialogo donde le solicita el archivo& \\
    \hline
    3. El PM Agrega al archivo y selecciona la opcion guardar & 3.1 El PM Selecciona Cancelar \newline 3.2 Continua en paso 1\\
    \hline
    4.El Sistema Agrega la nueva propuesta&\\
    \hline
    5.Fin de C.U. &\\
    \hline
\end{longtable}

% GERENTE
		\subsubsection{Detalles de Casos de Uso del Gerente}
\begin{longtable}{|p{.60\textwidth}|p{.40\textwidth}|}
    \hline
    \multicolumn{2}{|p{16cm}|}{
        \textbf{Caso de uso:}Asignando PM Al Proyecto\newline
        \textbf{Actor:} Gerente\newline
        \textbf{Pre: }Gerente Autenticado\newline
        \textbf{Post:} Un PM es asignado al proyecto
    }\\
    \hline
    1.El Gerente Consulta los proyectos nuevos en el sistema USA Consultando Nuevos Trabajos&    \\
    \hline
    2.El Gerente Consulta los mejores PM para el proyecto dado USA Consutlando TOP Proveedores& \\
    \hline
    3.El Gerente Asigna el mejor PM Al Proyecto&\\
    \hline
    4.Fin del C.U&\\
    \hline
\end{longtable}


\begin{longtable}{|p{.60\textwidth}|p{.40\textwidth}|}
    \hline
    \multicolumn{2}{|p{16cm}|}{
        \textbf{Caso de uso:}Consultando Reportes de proyecto\newline
        \textbf{Actor:} Gerente\newline
        \textbf{Pre: }true\newline
        \textbf{Post:}  El Gerente Consulta el estado de un proyecto
    }\\
    \hline
    1.El Gerente Busca un proyecto segun ciertos filtros. Extiende Consultando Proyectos&    \\
    \hline
    2.El Sistema le devuelve una lista de proyectos& \\
    \hline
    3.El Gerente Selecciona un proyecto y presiona el boton de obtener estado de proyecto&\\
    \hline
    4.El Sistema Devuelve el estado del proyecto seleccionado &\\
    \hline
    5.Fin del C.U.&\\
    \hline
\end{longtable}

\begin{longtable}{|p{.60\textwidth}|p{.40\textwidth}|}
    \hline
    \multicolumn{2}{|p{16cm}|}{
        \textbf{Caso de uso:}Consultando Nuevos Trabajos\newline
        \textbf{Actor:} Gerente\newline
        \textbf{Pre: }true\newline
        \textbf{Post:}  El Gerente Consulta Nuevos Trabajos
    }\\
    \hline
    1.El Selecciona la opcion ver nuevos trabajos&    \\
    \hline
    2.El Sistema le devuelve una lista de nuevos trabajos& \\
    \hline
    3.El Gerente Selecciona un trabajo y desea dar de alta un nuevo proyecto.Extiende Caso de uso Dando Alta Nuevo Proyecto\\
    \hline
    4.Fin del C.U.&\\
    \hline
\end{longtable}

\begin{longtable}{|p{.60\textwidth}|p{.40\textwidth}|}
    \hline
    \multicolumn{2}{|p{16cm}|}{
        \textbf{Caso de uso:}Consultando Nuevos Trabajos\newline
        \textbf{Actor:} Gerente\newline
        \textbf{Pre: }El Gerente consulto nuevos trabajos\newline
        \textbf{Post:}  El Gerente Consulta Nuevos Trabajos
    }\\
    \hline
    1.El Gerente Selecciona  la opcion dar de alta nuevos trabajos&    \\
    \hline
    2.El Sistema le consulta si desea agregar un PM al nuevo proyecto& \\
    \hline
    3.El Gerente selecciona la opcion SI. Extiende Caso de uso Consultando Lista de PM\\
    \hline
    4.El gerente Guarda los cambios. Usa Caso de uso Notificando Alta de Proyecto&\\
    \hline
    5.Fin C.U&\\
    \hline
\end{longtable}

\begin{longtable}{|p{.60\textwidth}|p{.40\textwidth}|}
    \hline
    \multicolumn{2}{|p{16cm}|}{
        \textbf{Caso de uso:}Consultando Nuevos Trabajos\newline
        \textbf{Actor:} Gerente\newline
        \textbf{Pre: }El Gerente fue notificado de un nuevo Contrato para Aprobar\newline
        \textbf{Post:}  El Gerente Aprueba un contrato de trabajo
    }\\
    \hline
    1.El Gerente Ingresa al Link en el mail donde fue notificado de un nuevo contrato a aprobar&    \\
    \hline
    2.El Sistema le muestra el contrato para aprobar,y ademas tiene la opcion de Aprobar y Rechazar& \\
    \hline
    3.El Gerente selecciona la opcion Aprobar. USA Caso de uso Notificando Contrato Aprobado/Rechazado& 3.1 El gerente Selecciona la opcion Rechazar. Usa Caso de uso Notificando Contrato Aprobado/Rechazado\\
    \hline
    4.Fin C.U&\\
    \hline
\end{longtable}

% Proveedor
		\subsubsection{Detalles de Casos de Uso del Proveedor}
\begin{longtable}{|p{.60\textwidth}|p{.40\textwidth}|}
    \hline
    \multicolumn{2}{|p{16cm}|}{
        \textbf{Caso de uso:}Respondiendo solicitud Datos/Seguro de caucion\newline
        \textbf{Actor:} Proveedor\newline
        \textbf{Pre: }El Proveedor recibio un mail solicitandosele los datos del seguro de caucion\newline
        \textbf{Post:}  El Proveedor Agrega los datos del seguro de caucion
    }\\
    \hline
    1.El Proveedor Ingresa al Link en el mail donde fue notificado&    \\
    \hline
    2.El Sistema le muestra una ventana de dialogo solicitandole un archivo con el seguro de caucion, y el boton de guardar.& \\
    \hline
    3.El Proveeodor adjunta el archivo del seguro de caucion y selecciona la opcion Guardar. &\\
    \hline
    4.El sistema guarda el seguro de caucion asociandoselo al proveedor. USA Caso de uso Notificando datos/seguro de caución completados al Admin&\\
    \hline
    5.Fin C.U&\\
    \hline
\end{longtable}

\begin{longtable}{|p{.60\textwidth}|p{.40\textwidth}|}
    \hline
    \multicolumn{2}{|p{16cm}|}{
        \textbf{Caso de uso:}Respondiendo solicitud de propuesta para proyecto\newline
        \textbf{Actor:} Proveedor\newline
        \textbf{Pre: }El Proveedor recibio un mail solicitandosele la propuesta del proyecto\newline
        \textbf{Post:}  El Proveedor envia la propuesta del proyecto
    }\\
    \hline
    1.El Proveedor Ingresa al Link en el mail donde fue notificado&    \\
    \hline
    2.El Sistema le muestra un formulario donde puede ingresar el detalle de los materiales y costos.Ademas puede adjuntar un archivo& \\
    \hline
    3.El Proveeodor completa el formulario y selecciona la opcion Guardar. &3.1 El Proveedor Completa el formulario y Adjunta un archivo.Luego selecciona la opcion guardar. Continua en paso 4\\
    \hline
    4.El sistema guarda la propuesta y la adjunta al proyecto correspondiente. USA Caso de uso Notificando propuesta enviada a PM&\\
    \hline
    5.Fin C.U&\\
    \hline
\end{longtable}

\begin{longtable}{|p{.60\textwidth}|p{.40\textwidth}|}
    \hline
    \multicolumn{2}{|p{16cm}|}{
        \textbf{Caso de uso:}Anotandose como proveedor\newline
        \textbf{Actor:} Proveedor\newline
        \textbf{Pre: }True\newline
        \textbf{Post:}  El Proveedor se anota como proveedor valido
    }\\
    \hline
    1.El Proveedor Ingresa a la pagina web de la empresa y selecciona la opcion agregarse como proveedor&    \\
    \hline
    2.La pagina web le muestra un formulario solicitando nombre, datos de ubicacion , telefono y mail como datos obligatorios.Ademas tiene un boton de enviar& \\
    \hline
    3.El Proveeodor completa los datos y selecciona la opcion Enviar. &\\
    \hline
    4.El sistema guarda los datos del proveedor. Notificando Nuevo proveedor al Admin&\\
    \hline
    5.Fin C.U&\\
    \hline
\end{longtable}


\section{Modelo conceptual}
\subsection{Condiciones OCL}
\begin{itemize}
		\item	Lugares en donde se pueden redefinir cosas:
	
			Context: Proyecto
			
			\begin{tabular}{ll}
				self.redefiniendoProveedor $\Rightarrow$	& BuscandoProveedor < self.Status $\leq$ ObraEnCurso	\\
				self.redefiniendoPM $\Rightarrow$			& EligiendoPM < self.Status $\leq$ ObraEnCurso			\\
			\end{tabular}

	\item Qu\'e tiene que pasar con el proyecto en cada estado:

			Context: Proyecto
			
			\begin{tabular}{ll}
				\laterThan{EligiendoPM}				& \notEmpty{self.supervisaHistorico} and						\\
													& ((self.redefiniendoPM) xor (\notEmpty{self.supervisaActual}))	\\
				
				and \laterThan{DefiniendoAlcance}	& \notEmpty{self.alcance} \\
				
				and \laterThan{BuscandoProveedor}	& \notEmpty{self.proveedorHistorico} and	\\
													& (self.redefiniendoProveedor xor			\\
													& (\notEmpty{self.proveedorActual}))		\\
				
				and \laterThan{FirmandoContratos}	& \notEmpty{contratoCliente(self, self.Solicitante)} and	\\
													& (self.redefiniendoProveedor xor							\\
													& \notEmpty{contratoProveedor(self, self.ProveedorActual)})	\\
				
				and \laterThan{ConsiguiendoFeedback}	& \notEmpty{self.FeedbackProveedor} and	\\
														& \notEmpty{self.FeedbackPM}			\\
			\end{tabular}
			
	\item Seguro de caución al día
	
			Context: Proyecto
			
			\notEmpty{self.ProveedorActual} and self.Status $\leq$ ObraEnCurso $\Rightarrow$
			
			\begin{tabular}{ll}
				\hfill
				& \notEmpty{self.ProveedorActual.Seguro} and	\\
				& self.ProveedorActual.Seguro.Vencimiento $\geq$ ContratoProveedor(self, self.ProveedorActual).FechaFin	\\
			\end{tabular}
				
	\item 
\end{itemize}


	\subsection{Diagramas de procesos}

En esta sección detallamos distintos sub-procesos del flujo general de la empresa, 
utilizando distintos diagramas y técnicas, según el caso. 

	\subsubsection{Flujo macro de un proyecto}
Primero mostramos un FSM que muestra a nivel macro todas las etapas por las 
que pasa un proyecto, desde su creación hasta su finalización. Varias transiciones 
del mismo diagrama serán detalladas luego con otros diagramas. 

\begin{figure}[H]
\centering
\includegraphics[width=0.8\linewidth]{diag/nuevos/fsm-proj.png}
\caption{FSM: Flujo general de un proyecto}
\label{fsm-proj}
\end{figure}

Hacemos algunas aclaraciones sobre algunas transiciones del diagrama.\\

Para \textbf{GerenteAsignaPM}, esto en el caso que se necesite cambiar al PM 
por diversos motivos:
\begin{itemize}
	\item El Gerente debe consultar el estado de los PMs en el sistema. 
	\item Luego debe seleccionar uno que considere acorde y asignarlo al proyecto 
	en cuestión. 
\end{itemize}

Para \textbf{PMDefineAlcance}, esto es en el caso que se necesite cambiar de proveedor:
\begin{itemize}
	\item El PM debe relevar lo que falta completar del alcance global 
	con el cliente. 
	\item Redefinir el alcance usado para la búsqueda del proveedor. 
	\item Cargar el nuevo alcance en el sistema para luego buscar el nuevo proveedor. 
\end{itemize}

A continuación, mostramos un diagrama de actividad que ejemplifica un posible 
escenario de acciones desde que se carga un pre-proyecto hasta que se elige un 
proveedor. 

\begin{figure}[H]
\centering
\includegraphics[width=0.8\linewidth]{diag/nuevos/act-busqprov.png}
\caption{Diagrama de actividad: Búsqueda de proveedor}
\label{act-busqprov}
\end{figure}

Las acciones destacadas se detallan luego usando FSM. 


	\subsubsection{Definir Alcance (PM y Cliente)}
El proceso de \textit{Definir Alcance} mencionado en el diagrama anterior 
se detalla mediante el siguiente FSM. 

Luego del mismo, queda definido el alcance del proyecto con el cliente. 
Potencialmente, el PM podrá luego definir nuevos alcances, para buscar un nuevo 
proveedor en caso de necesitar cambiarlo, y que entre todos estos alcances 
se logre cumplir el alcance original definido para el cliente. 

\begin{figure}[H]
\centering
\includegraphics[width=0.8\linewidth]{diag/nuevos/fsm-alcance.png}
\caption{FSM: Definición de alcance}
\label{fsm-alcance}
\end{figure}


		\subsubsection{Búsqueda de proveedor}
El proceso de \textit{Selección de proveedor} de diagrama \ref{act-busqprov} se detalla mediante 
los siguientes FSM. Este proceso tiene como salida si se pudo seleccionar de 
manera exitosa o no un proveedor adecuado para el alcance. En caso de fallar, ya sea 
por falta de respuestas o porque no hay proveedores aptos, se redefine la búsqueda y 
se vuelve a empezar. 

\begin{figure}[H]
\centering
\includegraphics[width=0.8\linewidth]{diag/nuevos/fsm-selprov.png}
\caption{FSM: Búsqueda de proveedor}
\label{fsm-selprov}
\end{figure}


		\subsubsection{Firma de contratos}
Una vez seleccionado el proveedor para cumplir con el alcance, se inicia 
el proceso de firma de contratos. El mismo se detalla en los siguientes FSM. 

\begin{figure}[H]
\centering
\includegraphics[width=0.8\linewidth]{diag/nuevos/fsm-firmcont1.png}
\caption{FSM: Firma de contratos (1)}
\label{fsm-firmcont1}
\end{figure}

\begin{figure}[H]
\centering
\includegraphics[width=0.8\linewidth]{diag/nuevos/fsm-firmcont2.png}
\caption{FSM: Firma de contratos (2)}
\label{fsm-firmcont2}
\end{figure}

		\subsubsection{Supervisión de la obra y manejo de problemas}

\begin{figure}[H]
\centering
\includegraphics[width=0.8\linewidth]{diag/nuevos/fsm-obra1.png}
\caption{FSM: Supervisión de la obra (1)}
\label{fsm-firmcont1}
\end{figure}

\begin{figure}[H]
\centering
\includegraphics[width=0.8\linewidth]{diag/nuevos/fsm-obra2.png}
\caption{FSM: Supervisión de la obra (1)}
\label{fsm-firmcont2}
\end{figure}

	\subsubsection{Mantenimiento de las bases de datos}
Como se ha mencionado con anterioridad, el Admin es un agente dedicado 
a mantener las bases de datos de proveedores y clientes actualizadas. 
El mismo puede enviar solicitudes de datos tanto a proveedores como clientes, 
para que los mismos actualicen sus datos. 

Al enviar una de estas solicitudes, 
el sistema genera un formulario (con una cierta vigencia programada) y envía 
un por mail un link a dicho formulario. 

Cuando un proveedor/cliente recibe este mail, podrá abrir el formulario 
donde cargar directamente sus nuevos datos. De haber algún problema, o en caso 
de que decida no usar el formulario, puede enviar un mail directamente al Admin 
con sus datos para que éste los actualice en el sistema. 

Mostramos ahora 2 diagramas de actividad que detallan los procesos en que 
el Admin envía una solicitud a un cliente y a un proveedor. 

\begin{figure}[H]
\centering
\includegraphics[width=0.8\linewidth]{diag/nuevos/da-admin.png}
\caption{Diagrama de actividad: Admin y base de datos}
\label{da-admin}
\end{figure}
